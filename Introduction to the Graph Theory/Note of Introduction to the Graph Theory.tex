\documentclass[12pt,a4paper]{book}
\usepackage[utf8]{inputenc}
\usepackage{titlesec}
\usepackage{geometry}
\usepackage{lipsum}
\usepackage{hyperref}
\usepackage{xcolor}
\usepackage[most,breakable]{tcolorbox}
\usepackage{setspace}
\usepackage{fancyhdr}
\usepackage{tikz}
\usetikzlibrary{arrows.meta,positioning}

\geometry{margin=1in}
\setstretch{1.25}

\definecolor{myblue}{RGB}{30,90,150}
\definecolor{mygray}{RGB}{240,240,240}

\hypersetup{
  colorlinks=true,
  linkcolor=myblue,
  urlcolor=myblue
}

\titleformat{\chapter}[display]
  {\normalfont\bfseries\Huge\color{myblue}}
  {\filleft\Large\chaptertitlename\ \thechapter}
  {2ex}
  {\titlerule\vspace{2ex}\filright}
  [\vspace{2ex}\titlerule]

\titleformat{\section}
  {\Large\bfseries\color{myblue}}
  {\thesection}{1em}{}

\pagestyle{fancy}
\fancyhf{}
\fancyhead[L]{\leftmark}
\fancyhead[R]{\thepage}

\newtcolorbox{notebox}[1][]{
  colback=mygray,
  colframe=myblue,
  fonttitle=\bfseries,
  title=#1,
  sharp corners,
  boxrule=1pt,
  breakable
}

\title{Note of \textit{Introduction to the Graph Theory}}
\author{Zhehao Yi}
\date{Sep 18. 2025}

\begin{document}

\maketitle
\tableofcontents
\newpage

\chapter{Chapter 1: Introduction}

\section{The Block diagram of a Communication System}

\section{Channel Characteristics}

\section{Summary of System-Analysis Techniques}

\section{Probablilistic Approaches to System Optimization}
\begin{notebox}[Summary]
some summary
\end{notebox}

\section{Summary}
\begin{notebox}[Thinking]
some thinking
\end{notebox}

\chapter{Chapter 2:}
\section{}
\begin{notebox}[Summary]
some summary
\end{notebox}

\section{}
\begin{notebox}[Concept]
some concept 
\end{notebox}

\section{}
\begin{notebox}[Thinking]
some thinking
\end{notebox}

\end{document}
